\documentstyle[epsf,epic,eepic,eepicemu]{article} \oddsidemargin=-5mm
\evensidemargin=-5mm\marginparwidth=.08in \marginparsep=.01in
\marginparpush=5pt\topmargin=-15mm\headheight=12pt
\headsep=25pt\footheight=12pt \footskip=30pt\textheight=25cm
\textwidth=17cm\columnsep=2mm
\columnseprule=1pt\parindent=15pt\parskip=2pt

\begin{document}
\begin{center}
\bf Semestralni projekt NI-PDP 2021/2022:\\[5mm]
    Paralelni algoritmus pro reseni problemu\\[5mm]
    Jaroslav Langer\\[2mm]
magisterske studijum, FIT CVUT, Thakurova 9, 160 00 Praha 6\\[2mm]
\today
\end{center}

\section{Definice problemu a popis sekvencniho algoritmu}

Problem resi nalezeni maximalniho bipartitniho souvisleho podgrafu. Neboli
podgraf daneho grafu, ktery je bipartitni (vsechny vrcholy lze obarvit dvema
barvami tak, ze kazde dva sousedni vrcholy maji jinou barvu), je souvisly
(kazdy vrchol je dosazitelny) a vaha hran je maximalni.

% Sekvencni algoritmus dostane cestu k souboru s grafem jako prvni argument. Ten
% nacte do pole struktur \verb|Edge|, ktera obsahuje vahu hrany prvni vrchol (ten
% s mensim indexem) a druhy vrchol. Toto pole je serazeno sestupne podle vahy
% hran, aby prorezavani probihalo co nejrychleji.
%
% Po nacteni grafu je otestovano, zda graf neni jiz bipartitini spojity - DFS
% pruchodem se navstivi vsechny vrcholy, ktere se postupne obarvuji. Pokud jsou
% po pruchodu vsechny vrcholy navstivene, potom jsme nasli reseni jehoz vaha je
% soucet vah vsech hran a dve disjunktni mnoziny vrcholu jsou definovane podle
% barev z DFS pruchodu.
%
% V pripade, ze graf neni bipartitni spojity, zavola se rekurzivni funkce
% \verb|solve|, ktera prijima argument \verb|state|


Jako vychozi pouzijte text zadani, ktery rozsirte o presne vymezeni vsech
odchylek, ktere jste vuci zadani behem implementace provedli (napr.  upravy
heuristicke funkce, organizace zasobniku, apod.). Zminte i pripadne i takove
prvky algoritmu, ktere v zadani nebyly specifikovany, ale ktere se ukazaly jako
dulezite.  Dale popiste vstupy a vystupy algoritmu (format vstupnich a
vystupnich dat). Uvedte tabulku namerenych casu sekvencniho algoritmu pro ruzne
velka data.

\section{Popis paralelniho algoritmu a jeho implementace v OpenMP - taskovy paralelismus}

Popiste paralelni algoritmus, opet vyjdete ze zadani a presne vymezte
odchylky, ktere pri implementaci OpenMP pouzivate.
Popiste a vysvetlete strukturu celkoveho
paralelniho algoritmu na urovni procesuu v OpenMP a strukturu kodu
jednotlivych procesu. Napr. jak je naimplemtovana smycka pro cinnost
procesu v aktivnim stavu i v stavu necinnosti. Jake jste zvolili
konstanty a parametry pro skalovani algoritmu. Struktura a semantika
prikazove radky pro spousteni programu.


\section{Popis paralelniho algoritmu a jeho implementace v OpenMP - datovy paralelismus}

Popiste paralelni algoritmus, opet vyjdete ze zadani a presne vymezte
odchylky, ktere pri implementaci OpenMP pouzivate.
Popiste a vysvetlete strukturu celkoveho
paralelniho algoritmu na urovni procesuu v OpenMP a strukturu kodu
jednotlivych procesu. Napr. jak je naimplemtovana smycka pro cinnost
procesu v aktivnim stavu i v stavu necinnosti. Jake jste zvolili
konstanty a parametry pro skalovani algoritmu. Struktura a semantika
prikazove radky pro spousteni programu.

\section{Popis paralelniho algoritmu a jeho implementace v MPI}

Popiste paralelni algoritmus, opet vyjdete ze zadani a presne vymezte
odchylky, zvlaste u Master-Slave casti. Popiste a vysvetlete strukturu celkoveho
paralelniho algoritmu na urovni procesuu v MPI a strukturu kodu
jednotlivych procesu. Napr. jak je naimplemtovana smycka pro cinnost
procesu v aktivnim stavu i v stavu necinnosti. Jake jste zvolili
konstanty a parametry pro skalovani algoritmu. Struktura a semantika
prikazove radky pro spousteni programu.

\section{Namerene vysledky a vyhodnoceni}

\begin{enumerate}
\item Zvolte tri instance problemu s takovou velikosti vstupnich dat, pro ktere ma sekvencni
algoritmus casovou slozitost alespon nekolik minut - vice informaci na {\tt http://courses.fit.cvut.cz} v sekci "Organizace cviceni".
Pro mereni cas potrebny na cteni dat z disku a ulozeni na disk neuvazujte a zakomentujte
ladici tisky, logy, zpravy a vystupy.
\item Merte paralelni cas pri pouziti $i=2,\cdot,60$ vypocetnich jader.
\item Tabulkova a pripadne graficky zpracovane namerene hodnoty casove slozitosti měernych instanci behu programu s popisem instanci dat. Z namerenych dat sestavte grafy zrychleni $S(n,p)$.
\item Analyza a hodnoceni vlastnosti paralelniho programu, zvlaste jeho efektivnosti a skalovatelnosti, pripadne popis zjisteneho superlinearniho zrychleni.

\end{enumerate}

\section{Zaver}

Celkove zhodnoceni semestralni prace a zkusenosti ziskanych behem semestru.

\section{Literatura}


\end{document}
